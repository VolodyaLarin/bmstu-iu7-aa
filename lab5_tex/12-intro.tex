\Introduction

\pagenumbering{arabic}
\setcounter{page}{2}


Для увеличения скорости выполнения программ используют параллельные вычисления. Конвейерная обработка данных является популярным приемом при работе с параллельностью. Она позволяет на каждой следующей «линии» конвейера использовать данные, полученные с предыдушего этапа.

Конвейер — способ организации вычислений, используемый в современных процессорах и контроллерах с целью повышения их производительности (эксплуатация параллелизма на уровне инструкций).

Целью данной лабораторной работы является изучение принципов конвейерной обработки данных.

Для достижения поставленной цели необходимо выполнить следующие задачи:

\begin{itemize}
	\item исследовать основы конвейерной обработки данных;
	\item привести схемы алгоритмов, используемых для конвейерной и линейной обработок данных;
	\item описать используемые структуры данных;
	\item описать структуру разрабатываемого ПО;
	\item определить средства программной реализации;
	\item провести сравнительный анализ времени работы алгоритмов;
	\item провести модульное тестирование;
	\item описать и обосновать полученные результаты в отчете о выполненной лабораторной работе.
\end{itemize}



