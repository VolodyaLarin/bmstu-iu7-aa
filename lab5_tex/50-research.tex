\chapter{Экспериментальный раздел}
\label{cha:research}

\section{Технические характеристики}
Тестирование выполнялось на устройстве со следующими техническими характеристиками:
\begin{itemize}
	\item Операционная система Ubuntu 21.04;
	\item Память 16 GiB (4,5 GiB выделено для нужд графического ядра)
	\item Процессор AMD® Ryzen 5 5500u with radeon graphics × 12 
\end{itemize}

\section{Время выполнения алгоритмов}

Для замеров времени использовалась стандартная функция языка \textit{std::chrono::system\_clock::now()}  \cite{clock}.  Данная функция возвращает точку времени, представляющую текущее время системы.  суммарное процессорное время, использованное программой. На листинге \ref{listing:mesuare} показан способ применения данной функции при замерах.

\begin{lstlisting}[caption={Замер времени функции}, label=listing:mesuare]
timeStart = std::chrono::system_clock::now();
stage1(std::ref(q1), std::ref(q2));
timeEnd = std::chrono::system_clock::now();
\end{lstlisting}

Из результатов работы программы для 5 матриц размером 1000, представленных на листинге \ref{lst:test_tbl}, можно сделать вывод, что конвейерная обработка работает согласно требованиям, выставленных в аналитической части.

\lstinputlisting[caption={Результаты работы программы для 5 матриц размером 1000},label=lst:test_tbl]{code/test.tbl}


На рисунке \ref{fig:plot_count} представлена зависимость времени работы обработки 100 матриц от их размеров. 



\begin{figure}[H]
    \centering

    \begin{tikzpicture}
        \begin{axis} [
            legend pos = north west,
            ymin = 0,
            xmin = 0,
            grid = major,
            xlabel={количество матриц},
            ylabel={Время работы, с},
			xtick={0,200,400,800,1600,3200},
            table/col sep = semicolon,
            /pgf/number format/1000 sep={}
        ]
        \legend{
            Конвейерная обработка,
            Линейная обработка,
        };
        \addplot table [x=x, y=y] {bench_results/conv_count.csv};
        \addplot table [x=x, y=y] {bench_results/lin_count.csv};
        \end{axis}
    \end{tikzpicture}

    \caption{Зависимость времени работы обработки матриц размера 100 от их количества}
    \label{fig:plot_count}
\end{figure}

На рисунке \ref{fig:plot_size} представлена зависимость времени работы обработки матриц размера 100 от их количества. 

\begin{figure}[H]
    \centering

    \begin{tikzpicture}
        \begin{axis} [
            legend pos = north west,
            ymin = 0,
            grid = major,
            xlabel={Размер матриц},
            ylabel={Время работы, с},
            table/col sep = semicolon,
            /pgf/number format/1000 sep={}
        ]
        \legend{
            Конвейерная обработка,
            Линейная обработка,
        };
        \addplot table [x=x, y=y] {bench_results/conv_size.csv};
        \addplot table [x=x, y=y] {bench_results/lin_size.csv};
        \end{axis}
    \end{tikzpicture}

    \caption{Зависимость времени работы обработки 100 матриц от их размеров}
    \label{fig:plot_size}
\end{figure}


\section{Вывод}

В этом разделе были указаны технические характеристики машины, на которой происходило сравнение времени работы алгоритмов обработки матриц для конвейерной и ленейной реализаций.

В результате замеров времени было установлено, что конвейерная реализация обработки лучше линейной
при большом количестве матриц (в 2.5 раза при 400 матрицах, в 2.6 раза при 800 и в 2.7 при 1600). Так же конвейерная обработка показала себя лучше при увеличении размеров обрабатываемых матриц (в 2.8 раза при размере матриц 160х160, в 2.9 раза при размере 320х320 и в 2.9 раза при матрицах 640х640). Значит при большом количестве обрабатываемых матриц, а так же при матрицах большого размера стоит использовать конвейерную реализацию обработки, а не линейную.



%%% Local Variables:
%%% mode: latex
%%% TeX-master: "rpz"
%%% End:
