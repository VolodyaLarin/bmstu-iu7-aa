\chapter{Экспериментальный раздел}
\label{cha:research}

\section{Технические характеристики}
Тестирование выполнялось на устройстве со следующими техническими характеристиками:
\begin{itemize}
	\item Операционная система Ubuntu 21.04;
	\item Память 16 GiB (4,5 GiB выделено для нужд графического ядра)
	\item Процессор AMD® Ryzen 5 5500u with radeon graphics × 12 
\end{itemize}

\section{Время выполнения алгоритмов}

Для замеров времени использовалась стандартная функция языка clock \cite{clock}.  Данная функция возвращает суммарное процессорное время, использованное программой. В случае ошибки, функция возвращает значение -1. На листинге \ref{listing:mesuare} показан способ применения данной функции при замерах.

\begin{lstlisting}[caption={Замер времени функции}, label=listing:mesuare]
auto start = clock();
for (int j = 0; j < counts; ++j) {
    rec_l(left, right);
}
res.rl = double(clock() - start) / counts;
\end{lstlisting}


Результаты тестирования приведены в таблице \ref{tbl:time}.

\begin{table}[H]
    \caption{Таблица зависимости времени работы реализаций алгоритмов от длины входных данных}
    \begin{tabular}{||l|lll||}
    \hline
    Длина массива & \begin{tabular}[c]{@{}l@{}} Пузырьком\end{tabular} & \begin{tabular}[c]{@{}l@{}}Вставками\end{tabular} & \begin{tabular}[c]{@{}l@{}}Выбором\end{tabular} \\ \hline \hline
    1          & 0.17                                                  & 0.42                                              & 0.39                                            \\ \hline
    2          & 0.16                                                  & 0.14                                              & 0.17                                            \\ \hline
    3          & 0.16                                                  & 0.16                                              & 0.19                                            \\ \hline
    4          & 0.19                                                  & 0.17                                              & 0.2                                             \\ \hline
    5          & 0.22                                                  & 0.18                                              & 0.24                                            \\ \hline
    6          & 0.27                                                  & 0.22                                              & 0.26                                            \\ \hline
    7          & 0.28                                                  & 0.22                                              & 0.3                                             \\ \hline
    8          & 0.34                                                  & 0.26                                              & 0.34                                            \\ \hline
    9          & 0.43                                                  & 0.31                                              & 0.38                                            \\ \hline
    10         & 0.53                                                  & 0.35                                              & 0.43                                            \\ \hline
    20         & 1.56                                                  & 0.88                                              & 1.2                                             \\ \hline
    30         & 3.67                                                  & 1.91                                              & 3.98                                            \\ \hline
    50         & 10.33                                                 & 4.47                                              & 6.6                                             \\ \hline
    100        & 38.3                                                  & 16.85                                             & 24.64                                           \\ \hline
    200        & 158.61                                                & 55.62                                             & 90.96                                           \\ \hline
    500        & 974.95                                                & 294.31                                            & 524.33                                          \\ \hline
    1000       & 4715.04                                               & 1558.64                                           & 2031.2                                          \\ \hline
    \end{tabular}
    \label{tbl:time}
    \end{table}

Представлены зависимости времени работы  для случайного набора данных представлены  на рисунках \ref{fig:plot_random} и \ref{fig:plot_random_big}, для отсортированного набора данных на рисунках \ref{fig:plot_sorted} и \ref{fig:plot_sorted_big}, для обратно отсортированного набора данных на рисунках \ref{fig:plot_resorted} и \ref{fig:plot_resorted_big} . 
\begin{figure}[H]
    \centering
    
    \begin{tikzpicture}
        \begin{axis} [
            legend pos = north west, 
            ymin = 0, 
            grid = major,
            xlabel={Длина массива},
            ylabel={Количество тиков},
            table/col sep = semicolon,
            /pgf/number format/1000 sep={}
        ]
        \legend{ 
            Сортировка пузырьком, 
            Сортировка вставками, 
            Сортировка выбором
        };
        \addplot table [x=x, y=y] {bench_results/bubble_random.csv};
        \addplot table [x=x, y=y] {bench_results/insert_random.csv};
        \addplot table [x=x, y=y] {bench_results/select_random.csv};
        \end{axis}
    \end{tikzpicture}

    \caption{Зависимость времени работы реализаций алгоритмов сортировки от длины массивов, содержащих случайный набор данных}
    \label{fig:plot_random}
\end{figure} 

Следует рассмотреть участок длинны массива от 1 до 10 более детально. Данная зависимость представлена на рисунке \ref{fig:plot_random_big}. На малых размерах длинны массива 1-2 результаты не имеет смысла, так как задачи такого рода не решаются с помощью сортировки, а с помощью простого обмена чисел.

\begin{figure}[H]
    \centering
    
    \begin{tikzpicture}
        \begin{axis} [
            legend pos = north west, 
            ymin = 0, 
            grid = major,
            xlabel={Длина массива},
            ylabel={Количество тиков},
            table/col sep = semicolon,
            /pgf/number format/1000 sep={}
        ]
        \legend{ 
            Сортировка пузырьком, 
            Сортировка вставками, 
            Сортировка выбором
        };
        \addplot table [x=x, y=y] {bench_results/bubble_random copy.csv};
        \addplot table [x=x, y=y] {bench_results/insert_random copy.csv};
        \addplot table [x=x, y=y] {bench_results/select_random copy.csv};
        \end{axis}
    \end{tikzpicture}

    \caption{Зависимость времени работы реализаций алгоритмов сортировки от длины массивов, содержащих случайный набор данных (крупно)}
    \label{fig:plot_random_big}
\end{figure} 

Представлены зависимости времени работы  для отсортированного набора данных представлены на рисунках \ref{fig:plot_sorted} и \ref{fig:plot_sorted_big}.

\begin{figure}[H]
    \centering
    
    \begin{tikzpicture}
        \begin{axis} [
            legend pos = north west, 
            ymin = 0, 
            grid = major,
            xlabel={Длина массива},
            ylabel={Количество тиков},
            table/col sep = semicolon,
            /pgf/number format/1000 sep={}
        ]
        \legend{ 
            Сортировка пузырьком, 
            Сортировка вставками, 
            Сортировка выбором
        };
        \addplot table [x=x, y=y] {bench_results/bubble_sorted.csv};
        \addplot table [x=x, y=y] {bench_results/insert_sorted.csv};
        \addplot table [x=x, y=y] {bench_results/select_sorted.csv};
        \end{axis}
    \end{tikzpicture}

    \caption{Зависимость времени работы реализаций алгоритмов сортировки от длины массивов, содержащих отсортированный набор данных}
    \label{fig:plot_sorted}
\end{figure} 



\begin{figure}[H]
    \centering
    
    \begin{tikzpicture}
        \begin{axis} [
            legend pos = north west, 
            ymin = 0, 
            grid = major,
            xlabel={Длина массива},
            ylabel={Количество тиков},
            table/col sep = semicolon,
            /pgf/number format/1000 sep={}
        ]
        \legend{ 
            Сортировка пузырьком, 
            Сортировка вставками, 
            Сортировка выбором
        };
        \addplot table [x=x, y=y] {bench_results/bubble_sorted copy.csv};
        \addplot table [x=x, y=y] {bench_results/insert_sorted copy.csv};
        \addplot table [x=x, y=y] {bench_results/select_sorted copy.csv};
        \end{axis}
    \end{tikzpicture}

    \caption{Зависимость времени работы реализаций алгоритмов сортировки от длины массивов, содержащих отсортированный набор данных (крупно)}
    \label{fig:plot_sorted_big}
\end{figure} 

Представлены зависимости времени работы  для обратно отсортированного набора данных представлены на рисунках \ref{fig:plot_resorted} и \ref{fig:plot_resorted_big}.



\begin{figure}[H]
    \centering
    
    \begin{tikzpicture}
        \begin{axis} [
            legend pos = north west, 
            ymin = 0, 
            grid = major,
            xlabel={Длина массива},
            ylabel={Количество тиков},
            table/col sep = semicolon,
            /pgf/number format/1000 sep={}
        ]
        \legend{ 
            Сортировка пузырьком, 
            Сортировка вставками, 
            Сортировка выбором
        };
        \addplot table [x=x, y=y] {bench_results/bubble_resorted.csv};
        \addplot table [x=x, y=y] {bench_results/insert_resorted.csv};
        \addplot table [x=x, y=y] {bench_results/select_resorted.csv};
        \end{axis}
    \end{tikzpicture}

    \caption{Зависимость времени работы реализаций алгоритмов сортировки от длины массивов, содержащих обратно отсортированный набор данных}
    \label{fig:plot_resorted}
\end{figure} 


\begin{figure}[H]
    \centering
    
    \begin{tikzpicture}
        \begin{axis} [
            legend pos = north west, 
            ymin = 0, 
            grid = major,
            xlabel={Длина массива},
            ylabel={Количество тиков},
            table/col sep = semicolon,
            /pgf/number format/1000 sep={}
        ]
        \legend{ 
            Сортировка пузырьком, 
            Сортировка вставками, 
            Сортировка выбором
        };
        \addplot table [x=x, y=y] {bench_results/bubble_resorted copy.csv};
        \addplot table [x=x, y=y] {bench_results/insert_resorted copy.csv};
        \addplot table [x=x, y=y] {bench_results/select_resorted copy.csv};
        \end{axis}
    \end{tikzpicture}

    \caption{Зависимость времени работы реализаций алгоритмов сортировки от длины массивов, содержащих обратно отсортированный набор данных (крупно)}
    \label{fig:plot_resorted_big}
\end{figure} 


\clearpage 
\section{Вывод}

В результате исследования было установлено, что сортировка вставками работает быстрее сортировки пузырьком в 3 раза и быстрее сортировки выбором в 1,3 раза на всех наборах данных.


%%% Local Variables:
%%% mode: latex
%%% TeX-master: "rpz"
%%% End:
