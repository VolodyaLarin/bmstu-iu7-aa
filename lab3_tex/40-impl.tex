\chapter{Технологический раздел}
\label{cha:impl}


\section{Средства реализации}


Основным средством разработки является язык программирования. Был выбран язык программирования C++. Данный выбор обоснован высокой скоростью работы языка, поддержкой строгой типизации \cite{cpplang}. Для сборки проекта был выбран инструмент CMake \cite{cmake}.  В качестве среды разработки был выбран инструмент JetBrains Clion \cite{clion}.


\section{Реализация алгоритмов}

В листинге \ref{lst:bubble} представлена реализация сортировки пузырьком.
В листинге \ref{lst:insert} представлена реализация сортировки вставками.
В листинге \ref{lst:select} представлена реализация сортировки выбором.

\lstinputlisting[language=c,caption={Алгоритм сортировки пузырьком},label=lst:bubble]{code/bubble.cpp}
\lstinputlisting[language=c,caption={Алгоритм сортировки вставками},label=lst:insert]{code/insert.cpp}
\lstinputlisting[language=c,caption={Алгоритм сортировки выбором},label=lst:select]{code/select.cpp}

\section{Тестовые данные}

Для генерации тестовых данных были использованы функции представленные на листинге \ref{lst:generator}.

\lstinputlisting[language=c,caption={Генераторы тестовых данных},label=lst:generator]{code/generator.cpp}


\section{Вывод}

На основе схем из конструкторского раздела были разработаны реализации требуемых алгоритмов.
