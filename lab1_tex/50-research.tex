\chapter{Экспериментальный раздел}
\label{cha:research}

\section{Технические характеристики}
Тестирование выполнялось на устройстве со следующими техническими характеристиками:
\begin{itemize}
	\item Операционная система Ubuntu 21.04;
	\item Память 16 GiB (4,5 GiB выделено для нужд графического ядра)
	\item Процессор AMD® Ryzen 5 5500u with radeon graphics × 12 
\end{itemize}

\section{Время выполнения алгоритмов}

Для замеров времени использовалась стандартная функция языка clock \cite{clock}.  Данная функция возвращает суммарное процессорное время, использованное программой. В случае ошибки, функция возвращает значение -1. На листинге \ref{listing:mesuare} показан способ применения данной функции при замерах.

\begin{lstlisting}[caption={Замер времени функции}, label=listing:mesuare]
auto start = clock();
for (int j = 0; j < counts; ++j) {
    rec_l(left, right);
}
res.rl = double(clock() - start) / counts;
\end{lstlisting}


Результаты тестирования приведены в таблице \ref{tbl:time} .Прочерк в таблице означает что тестирование для этого набора данных не выполнялось.

\begin{table}[h]
    \caption{Таблица зависимости времени работы реализаций алгоритмов от длины входных слов}
    \begin{tabular}{||l|llll||}
    \hline
    Длина слов & \begin{tabular}[c]{@{}l@{}}Рекурсивный\\ Левенштейна\end{tabular} & \begin{tabular}[c]{@{}l@{}}Итеративный\\ Левенштейна\end{tabular} & \begin{tabular}[c]{@{}l@{}}Рекурсивный с кэшем\\ Левенштейна\end{tabular} & \begin{tabular}[c]{@{}l@{}}Рекурсивный\\ Дамерау--Левенштейна\end{tabular} \\ \hline \hline
    1          & 0.5                                                               & 0.3                                                               & 0.5                                                                       & 0.2                                                                        \\ \hline
    2          & 0.3                                                               & 0.4                                                               & 0.5                                                                       & 0.3                                                                        \\ \hline
    3          & 1.1                                                               & 0.6                                                               & 0.7                                                                       & 1.2                                                                        \\ \hline
    4          & 5.4                                                               & 0.8                                                               & 1.1                                                                       & 5.7                                                                        \\ \hline
    5          & 28                                                                & 1                                                                 & 1.5                                                                       & 30.8                                                                       \\ \hline
    6          & 149.9                                                             & 1.3                                                               & 2.1                                                                       & 173                                                                        \\ \hline
    7          & 800.4                                                             & 1.7                                                               & 3.2                                                                       & 906.2                                                                      \\ \hline
    8          & 4357.9                                                            & 2.8                                                               & 6.7                                                                       & 5021.7                                                                     \\ \hline
    9          & 23820.5                                                           & 2.4                                                               & 4.9                                                                       & 27907.9                                                                    \\ \hline
    10         & 132888                                                            & 3                                                                 & 5.9                                                                       & 146318                                                                     \\ \hline
    20         & --                                                                & 10.3                                                              & 22.5                                                                      & --                                                                         \\ \hline
    30         & --                                                                & 22.5                                                              & 46.1                                                                      & --                                                                         \\ \hline
    50         & --                                                                & 59.9                                                              & 132.8                                                                     & --                                                                         \\ \hline
    100        & --                                                                & 236.4                                                             & 530.1                                                                     & --                                                                         \\ \hline
    200        & --                                                                & 921                                                               & 2047.2                                                                    & --                                                                         \\ \hline
    \end{tabular}
    \label{tbl:time}
    \end{table}

Представлены зависимости времени работы рекурсивных алгоритмов Левенштейна и Дамерау--Левенштейна на рисунках \ref{fig:plot_rec_l} и \ref{fig:plot_rec_e}

\begin{figure}[h]
    \centering
    
    \begin{tikzpicture}
        \begin{axis} [
            legend pos = north west, 
            ymin = 0, 
            grid = major,
            xlabel={Длина слов},
            ylabel={Количество тиков},
            table/col sep = semicolon,
            /pgf/number format/1000 sep={}
        ]
        \legend{ 
            Рекурсивный алгоритм Дамерау-Левенштейна, 
            Рекурсивный алгоритм Левенштейна
        };
        \addplot table [x=x, y=y] {bench_results/dl.csv};
        \addplot table [x=x, y=y] {bench_results/rl.csv};
        \end{axis}
    \end{tikzpicture}

    \caption{Зависимость времени работы реализаций рекурсивных алгоритмов Левенштейна и Дамерау-Левенштейна от времени}
    \label{fig:plot_rec_l}
\end{figure} 


\begin{figure}[h]
    \centering
    
    \begin{tikzpicture}
        \begin{semilogyaxis} [
            legend pos = north west, 
            ymin = 0,
            grid = major,
            xlabel={Длина слов},
            ylabel={Количество тиков},
            table/col sep = semicolon,
            /pgf/number format/1000 sep={}
        ]
        \legend{ 
            Рекурсивный алгоритм Дамерау-Левенштейна, 
            Рекурсивный алгоритм Левенштейна
        };
        \addplot table [x=x, y=y] {bench_results/dl.csv};
        \addplot table [x=x, y=y] {bench_results/rl.csv};
        \end{semilogyaxis}
    \end{tikzpicture}

    \caption{Зависимость времени работы реализаций рекурсивных алгоритмов Левенштейна и Дамерау-Левенштейна от времени в логарифмической шкале}
    
    \label{fig:plot_rec_e}
\end{figure} 

На рисунках \ref{fig:plot_iter_1} и \ref{fig:plot_iter_2} представлена зависимость времени работы реализаций алгоритмов Левенштейна итеративного и рекурсивного с кэшем. 

\begin{figure}[h]
    \centering
    
    \begin{tikzpicture}
        \begin{axis} [
            legend pos = north west, 
            ymin = 0, 
            grid = major,
            xlabel={Длина слов},
            ylabel={Количество тиков},
            table/col sep = semicolon,
            /pgf/number format/1000 sep={}
        ]
        \legend{ 
            Итеративный алгоритм Левенштейна, 
            Рекурсивный алгоритм Левенштейна с кэшированием
        };
        \addplot table [x=x, y=y] {bench_results/il.csv};
        \addplot table [x=x, y=y] {bench_results/rcl.csv};
        \end{axis}
    \end{tikzpicture}

    \caption{Зависимость времени работы реализаций алгоритмов Левенштейна итеративного и рекурсивного с кэшем}
    
    \label{fig:plot_iter_1}
\end{figure} 


\begin{figure}[h]
    \centering
    
    \begin{tikzpicture}
        \begin{axis} [
            legend pos = north west, 
            ymin = 0, 
            grid = major,
            xlabel={Длина слов},
            ylabel={Количество тиков},
            table/col sep = semicolon,
            /pgf/number format/1000 sep={}
        ]
        \legend{ 
            Итеративный алгоритм Левенштейна, 
            Рекурсивный алгоритм Левенштейна с кэшированием
        };
        \addplot table [x=x, y=y] {bench_results/il2.csv};
        \addplot table [x=x, y=y] {bench_results/rcl2.csv};
        \end{axis}
    \end{tikzpicture}

    \caption{Зависимость времени работы реализаций алгоритмов Левенштейна итеративного и рекурсивного с кэшем}
    
    \label{fig:plot_iter_2}
\end{figure} 

\section{Использование памяти}

Максимальная глубина стека при вызове рекурсивных функций рассчитывается по формуле \eqref{eq:rec-mem}.

\begin{eqndesc}
\begin{equation}\label{eq:rec-mem}
	M_{recursive} = (n \cdot lvar + ret + ret_{int}) \cdot depth
\end{equation}
Где:
\[
\begin{array}{l}
	n\text{ -- количество аллоцированных локальных переменных}; \\
	lvar\text{ -- размер переменной типа int} \\
	ret \text{ -- адрес возврата;}\\
	ret_{int} \text{ -- возвращаемое значение;}\\
	depth\text{ --  максимальная глубина стека вызова, которая равна} |S_1| + |S_2| .\\
\end{array}
\]
\end{eqndesc}

Использование памяти при итеративной реализации алгоритма Левенштейна может быть найдено с помощью формулы \eqref{eq:rec-mem-mem}.
\begin{eqndesc}
    \begin{equation}
        \label{eq:rec-mem-mem}
        M_{iter} = |S_1| + |S_2| + (|S_2| + 1) \cdot 2 \cdot lvar + n \cdot lvar + ret + ret_{int}
    \end{equation}
    Где $(|S_2| + 1) \cdot 2 \cdot lvar$ -- место в памяти под матрицу расстояний.
\end{eqndesc}

\clearpage 
\section{Вывод}
Рекурсивный алгоритм Левенштейна работает дольше итеративной реализации -- время этого алгоритма увеличивается экспонентально с ростом размера строк.
Рекурсивный алгоритм с кэшированием превосходит простой рекурсивный алгоритм по времени. 
Расстояние Дамерау -- Левенштейна по результатом замеров работает дольше в отличии от расстояния Левенштейна. Однако, в системах автоматического исправления текста, где транспозиция встречается чаще, расстояние Дамерау -- Левенштейна будет наиболее эффективным алгоритмом. 
По расходу памяти все реализации проигрывают рекурсивной за счет большого количества выделенной памяти под матрицу расстояний.


%%% Local Variables:
%%% mode: latex
%%% TeX-master: "rpz"
%%% End:
