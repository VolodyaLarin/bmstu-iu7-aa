\Introduction

\pagenumbering{arabic}
\setcounter{page}{2}

Расстояние Левенштейна - минимальное количество операций вставки одного символа, удаления одного символа и замены одного символа на другой, необходимых для превращения одной строки в другую.
\newline

Расстояние Левенштейна применяется в теории информации и компьютерной лингвистике для:
\begin{itemize}
	\item исправления ошибок в слове
	\item сравнения текстовых файлов утилитой diff
	\item в биоинформатике для сравнения генов, хромосом и белков
\end{itemize}

Цель данной лабораторной работы: 
\begin{enumerate}
	\item Изучение метода динамического программирования на материале алгоритмов нахождения расстояния Левенштейна и Дамерау-Левенштейна.
	\item Оценка реализаций алгоритмов нахождения расстояния Левенштейна и Дамерау-Левенштейна.
\end{enumerate}

Для достижения данных целей были выделены следущие задачи:
\begin{enumerate}
  	\item Изучение алгоритмов Левенштейна и Дамерау-Левенштейна;
	\item Применение метода динамического программирования для матричной реализации указанных алгоритмов; 
	\item Получение практических навыков реализации указанных алгоритмов: матричные и рекурсивные версии; 
	\item Сравнительный анализ линейной и рекурсивной реализаций выбранного алгоритма определения расстояния между строками по затрачиваемым ресурсам (времени и памяти); 
	\item Экспериментальное подтверждение различий во временнóй эффективности рекурсивной и
нерекурсивной реализаций выбранного алгоритма; 
	\item Описание и обоснование полученных результатов в отчете о выполненной лабораторной
работе, выполненного как расчётно-пояснительная записка к работе. 
\end{enumerate}

