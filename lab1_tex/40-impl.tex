\chapter{Технологический раздел}
\label{cha:impl}


\section{Требование к ПО}

Программа должна отвечать следующем требованиям:
\begin{enumerate}
	\item На вход подаются две строки в любой раскладке;
	\item ПО должно выводить полученное расстояние.
\end{enumerate}

\section{Средства реализации}


Основным средством разработки является язык программирования. Был выбран язык программирования C++. Данный выбор обоснован высокой скоростью работы языка, поддержкой строгой типизации \cite{cpplang}. Для сборки проекта был выбран инструмент CMake \cite{cmake}.  В качестве среды разработки был выбран инструмент JetBrains Clion \cite{clion}.

Для подгрузки тестовых данных в формате JSON \cite{json} в программу была использован модуль PropertyTree библиотеки Boost \cite{boost}.  

\section{Реализация алгоритмов}

В листинге \ref{lst:rl} представлена реализация рекурсивного алгоритма Левенштейна.
В листинге \ref{lst:rcl} представлена реализация рекурсивного алгоритма Левенштейна с кэшированием.
В листинге \ref{lst:il} представлена реализация итеративного алгоритма Левенштейна.
В листинге \ref{lst:dl} представлена реализация рекурсивного алгоритма Дамерау--Левенштейна.

\lstinputlisting[language=c,caption={Рекурсивный алгоритм Левенштейна},label=lst:rl]{code/rl.cpp}
\lstinputlisting[language=c,caption={Рекурсивный алгоритм Левенштейна с кэшированием},label=lst:rcl]{code/rcl.cpp}
\lstinputlisting[language=c,caption={Итеративный алгоритм Левенштейна},label=lst:il]{code/il.cpp}
\lstinputlisting[language=c,caption={Рекурсивный алгоритм Дамерау--Левенштейна},label=lst:dl]{code/dl.cpp}

\section{Тестовые данные}

В таблице \ref{tbl:tests} представлены тестовые данные для алгоритмов Левенштейна и Дамерау--Левенштейна. 

\begin{table}[h!]
        \caption{Тестовые данные для алгоритмов Левенштейна и Дамерау--Левенштейна}        
		\begin{tabular}{||c c c | c c ||} 
			\hline
			№ & $S_1$ & $S_2$ & Левенштейна & Дамерау--Левенштейна \\ [0.5ex] 
			\hline\hline
			1 & << >> & << >> & 0 & 0 \\ 
			2 & <<same>> & <<same>> & 0 & 0 \\ 
			3 & <<hello>> & <<gol>> & 4 & 4 \\ 
			4 & <<qwer>> & <<rewq>> & 4 & 3 \\ 
			5 & <<music>> & <<muisc>> & 2 & 1 \\
			6 & <<memory>> & <<memxory>> & 1 & 1 \\   
			7 & <<memxory>> & <<memory>> & 1 & 1 \\   
			8 & <<mexory>> & <<memory>> & 1 & 1 \\   
			\hline
		\end{tabular}
        \label{tbl:tests}
\end{table}

\section{Вывод}

На основе схем из конструкторского раздела были разработаны реализации требуемых алгоритмов.
