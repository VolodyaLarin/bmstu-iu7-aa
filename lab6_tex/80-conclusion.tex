\Conclusion % заключение к отчёту


Было экспериментально подтверждено различие во временной эффективности муравьиного алгоритма и алгоритма полного перебора решения задачи коммивояжера. В результате исследований можно сделать вывод о том, что при матрицах большого размера (больше 9) стоит использовать муравьиный алгоритм решения задачи коммивояжера, а не алгоритм полного перебора (на матрице размером 10x10 он работает в 15.4 раза быстрее). Также было установлено по результатам параметризации на экспериментальных класса данных, что при коэффиценте $\alpha$ = 0.1, 0.2, 0.3 муравьиный алгоритм работает наилучшим образом.
\vspace{5mm}

В ходе выполнения данной лабораторной работы были решены следующие задачи:
\begin{itemize}
	\item изучены алгоритмы решения задачи коммивояжера (муравьиный алгоритм и алгоритм полного перебора);
	\item применены изученные основы для реализации описанных алгоритмов;
	\item произведен сравнительный анализ муравьиного алгоритма и алгорита полного перебора для решения задачи коммивояжера;
	\item экспериментально подтверждено различие во временнoй эффективности рассматриваемых алгоритмов при помощи разработанного программного обеспечения на материале замеров процессорного времени;
	\item проведена параметризация муравьиного алгоритма, которая показала лучшие наборы параметров для работы алгоритма на экспериментальных классах данных;
	\item описаны и обоснованы полученные результаты в отчете о выполненной лабораторной работе.
\end{itemize}

Поставленная цель была достигнута.



%%% Local Variables: 
%%% mode: latex
%%% TeX-master: "rpz"
%%% End: 
