\Introduction

\pagenumbering{arabic}
\setcounter{page}{2}


Задача поиска оптимальных маршрутов является одной из важных.
Муравьиный алгоритм – один из эффективных полиномиальных алгоритмов для нахождения приближённых решений задачи коммивояжёра, а также решения аналогичных задач поиска маршрутов на графах. Суть подхода заключается в анализе и использовании модели поведения муравьёв, ищущих пути от колонии к источнику питания, и представляет собой метаэвристическую оптимизацию.

Целью данной лабораторной работы является изучение муравьиного алгоритма на примере задачи коммивояжера.

Для достижения поставленной цели необходимо выполнить следующие задачи:

\begin{itemize}
	\item исследовать задачу коммивояжера;
	\item изучить алгоритм полного перебора и муравьиный алгоритм для решения задачи коммивояжера;
	\item провести параметризацию муравьиного алгоритма на двух классах данных;
	\item привести схемы используемых алгоритмов;
	\item описать используемые структуры данных;
	\item описать структуру разрабатываемого ПО;
	\item определить средства программной реализации;
	\item провести сравнительный анализ времени работы алгоритмов;
	\item провести функциональное тестирование;
	\item описать и обосновать полученные результаты в отчете о выполненной лабораторной работе.
\end{itemize}





